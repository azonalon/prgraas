\documentclass[11pt]{article}
\usepackage[utf8]{inputenc}
\usepackage[english, ngerman]{babel}
\usepackage{amsmath,amsthm,verbatim,amssymb,amsfonts,amscd}
\usepackage{enumerate}
\usepackage{listings}
\usepackage{courier}
\lstset{
  numbers=left,
  language=C,
  basicstyle=\footnotesize\ttfamily,
  breaklines=true,
}
\newcommand{\abs}[1]{\left| #1 \right| }


\author{Felix Schrader, 3053850 \\
      Eduard Sauter, 3053470 }
\title{Programmieren I: Haus\"ubung 3}
\begin{document}
\maketitle
\subsection*{Aufgabe 1}
\begin{enumerate}[a)]
  \item $ $ % 
    \begin{lstlisting}
      x = 4 + 3 * (y = 1 + (z= 3*2 + 1))
      x = 4 + 3 * (y = 1 + (z=   6 + 1))
      x = 4 + 3 * (y = 1 + (z=       7))
      x = 4 + 3 * (y = 1 + (         7))
      x = 4 + 3 * (y = 1 +            7)
      x = 4 + 3 * (y =                8)
      x = 4 + 3 * (                   8)
      x = 4 + 3 * 8
      x = 4 + 24                         
      x = 28                            
      28                            
    \end{lstlisting}
    %
  \item $ $
    %
    \begin{lstlisting}
      8 + 7 <= 16^(2*7 + 3 == 17)
      15    <= 16^(2*7 + 3 == 17)
                1^(2*7 + 3 == 17)
                1^(14  + 3 == 17)
                1^(     17 == 17)
                1^(            1)
                1^1               
                0                 
    \end{lstlisting}
    %
  \item $ $
    %
    \begin{lstlisting}
      5 + 1 > 2*2 && (x = 7*2 == 28 / 7.0 ) || 1
      5 + 1 > 4   && (x = 7*2 == 28 / 7.0 ) || 1
      6     > 4   && (x = 7*2 == 28 / 7.0 ) || 1
                1 && (x = 7*2 == 28 / 7.0 ) || 1
                1 && (x = 14  == 28 / 7.0 ) || 1
                1 && (x = 14  == 4.       ) || 1
                1 && (x = 0               ) || 1
                1 && (0                   ) || 1
                1 && 0                      || 1
                0                           || 1
                1
    \end{lstlisting}
  
  
\end{enumerate} 

\subsection*{Aufgabe 4}
\begin{enumerate}[a)]
  \item
    Die Konstruktorfunktion ist der Erzeuger f\"ur Instanzen vom Typ
    der Itemization. Man sollte sie benutzen um die Korrektheit der Instanzen
    zu garantieren.
  \item
    Warum hei\ss{}t das konzept itemization?
    
\end{enumerate}


\end{document}
