\documentclass[11pt]{article}
\usepackage[utf8]{inputenc}
\usepackage[english, ngerman]{babel}
\usepackage{amsmath,amsthm,verbatim,amssymb,amsfonts,amscd}
\usepackage{enumerate}
\usepackage{listings}
\usepackage{courier}
\lstset{
  numbers=left,
  language=C,
  basicstyle=\footnotesize\ttfamily,
  breaklines=true,
}
\newcommand{\abs}[1]{\left| #1 \right| }


\author{Felix Schrader, 3053850 \\
      Eduard Sauter, 3053470 }
\title{Programmieren I: Haus\"ubung 2}
\begin{document}
\maketitle
\subsection*{Aufgabe 3: Skript}
\begin{enumerate}[a)]
  \item Falls Funktionen Nebeneffekte besitzen, so werden auch Variablen
    modifiziert, welche nicht als Argument der Funktion spezifiziert waren.
    Man kann also nicht mehr annehmen, dass eine Variable nach berechnung 
    der Funktion einen bestimmten Wert beh\"alt. Ein algebraischer
    Ausdruck kann keine Zuweisungen beinhalten. Ein Funktionsname in der 
    Programmiersprache C erf\"ullt diese jedoch Eigenschaft nicht. Deswegen sind
    algebraische Berechnungen mit C Funktionen nicht immer m\"oglich.

  \item In switch-statements darf nach case nur eine Konstante stehen, 
    kein beliebiger Ausdruck. Jedes \texttt{case} Statement muss mit
    \texttt{break} beendet werden wenn man nicht will, dass die anderen F\"alle
    auch \"uberpr\"uft werden. Anders formuliert: Die f\"alle schlie\ss{}en
    sich einander nicht gegenseitig aus.  Man neigt dazu, dies zu vergessen.
    Das beinhaltet auch den Fall, dass keiner der F\"alle eintritt. Man
    tendiert hierbei zu vergessen dass die M\"oglichkeit besteht, dieser Fall
    auch abgedeckt werden muss.

  \item Mit \texttt{enum} kann man Intervalle implementieren. Enums verhalten
    sich im Prinzip genauso wie als \texttt{const} deklarierte Variablen.

  \item Wir halten das Konzept von Intervallen in Programmiersprachen f\"ur
    interessant. Es ist h\"aufig ein Problem herauszufinden, ob man \"uber Intervalle,
    welche man benutzt, offen iterieren soll oder zum Beispiel den letzten
    Wert noch mit einschlie\ss{}t. Dies hat uns h\"aufig Probleme bereitet
    wenn Grenzf\"alle korrekt berechnet werden sollen.
\end{enumerate}

\end{document}
