\documentclass[11pt]{article}
\usepackage[utf8]{inputenc}
\usepackage[english, ngerman]{babel}
\usepackage{amsmath,amsthm,verbatim,amssymb,amsfonts,amscd}
\usepackage{enumerate}
\usepackage{listings}
\usepackage{courier}
\lstset{
  numbers=left,
  language=C,
  basicstyle=\footnotesize\ttfamily,
  breaklines=true,
}
\newcommand{\abs}[1]{\left| #1 \right| }


\author{Felix Schrader, 3053850 \\
      Eduard Sauter, 3053470 }
\title{Programmieren I: Haus\"ubung 4}
\begin{document}
\maketitle
\subsection*{Aufgabe 1}
\begin{enumerate}[a)]
  \item Man muss jedes Feld der Ausgabe einzeln auf Korrektheit \"uberpr\"ufen.

  \item Nein, derselbe Name darf nicht in beiden Strukturen vorkommen, 
    da sie den gleichen Namespace besetzen. Man kann allerdings den
    anonymen Strukturen einen unterscheidbaren Namen geben um diese Kollision
    zu vermeiden.

  \item Eine Schleife kann immer durch eine Rekursion ersetzt werden. 
    Betrachte dazu den folgenden Code:
\begin{lstlisting}
  
void body() {
    // execute stuff
}

void while_loop(bool cond) {
    body();
    if(cond)
        while_loop(cond);
    return;
}
\end{lstlisting}
  \item
    Wozu braucht man anonymous Structs oder Unions.
    
\end{enumerate} 

\subsection*{Aufgabe 4}
In dem Programm gibt es fünf Fehler.

Der erste Fehler ist, dass die \texttt{main}-Funktion groß geschrieben wurde.
Das ist ein Syntax-Fehler und das Programm kompiliert nicht.

Der zweite Fehler ist, dass bei dem String, der ausgegeben werden soll die
Anführungszeichen fehlen. \texttt{printf} soll also nicht definierte Variablen
ausgeben, auch das führt dazu, dass das Programm nicht kompiliert.

In dem String befindet sich der Fehler, dass statt \texttt{$\backslash$n}
\texttt{/n} geschrieben wurde. Dadurch wird statt einem Zeilenumbruch
\texttt{/n} angezeigt. Dies ist allerdings kein Syntax-Fehler, das Programm
kompiliert also trotzdem, es funktioniert nur nicht wie gewollt.

Am Ende des Funktionsausrufs \texttt{printf} muss ein Semikolon, wie nach jeder
Anweisung in C. Das fehlende Semikolon ist ein Syntaxfehler.

Der letzte Fehler ist, dass die geschwungene Klammer nach der
\texttt{main}-Funktion nicht wieder geschlossen wurde, auch hier kompiliert das
Programm nicht.

\end{document}
