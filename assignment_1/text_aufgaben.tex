\documentclass[11pt]{article}
\usepackage[utf8]{inputenc}
\usepackage[english, ngerman]{babel}
\usepackage{amsmath,amsthm,verbatim,amssymb,amsfonts,amscd}
\usepackage{enumerate}
\usepackage{listings}
\usepackage{courier}
\lstset{
  numbers=left,
  language=C,
  basicstyle=\footnotesize\ttfamily,
  breaklines=true,
  morekeywords={function}
}
\newcommand{\abs}[1]{\left| #1 \right| }


\author{Felix Schrader, Jens Duffert, Eduard Sauter}
\title{Datenstrukturen und Algorithmen: Haus\"ubung 1}
\begin{document}

\subsection*{Aufgabe 1}
\begin{enumerate}[i)]

\item Der Algorithmus sortiert die Zahlen einer Reihe in aufsteigender Reihenfolge.

\begin{table}[h!]
\centering
\begin{tabular}{|c|c|c|}
\hline 
Schrittnr. & Zustand vor Schritt & Algorithmus-Schritt \\ 
\hline 
1 & 11, 5, 1 & 11 wird ausgew\"ahlt \\ 
\hline 
2 & 11, 5, 1 & mit 5 vergleichen \\ 
\hline 
3 & 11, 5, 1 & $11>5\curvearrowright$ vertauschen und zur\"uck zum Anfang \\ 
\hline 
4 & 5, 11, 1 & 5 wird ausgew\"ahlt \\ 
\hline 
5 & 5, 11, 1 & mit 11 vergleichen \\ 
\hline 
6 & 5, 11, 1 & $5<11\curvearrowright$ n\"achste Zahl \\ 
\hline 
7 & 5, 11, 1 & 11 wird ausgew\"ahlt \\ 
\hline 
8 & 5, 11, 1 & mit 1 vergleichen \\ 
\hline 
9 & 5, 11, 1 & $11>1\curvearrowright$ vertauschen und zur\"uck zum Anfang \\ 
\hline 
10 & 5, 1, 11 & 5 wird ausgew\"ahlt \\ 
\hline 
11 & 5, 1, 11 & mit 1 vergleichen \\ 
\hline 
12 & 5, 1, 11 & $5>1\curvearrowright$ vertauschen und zur\"uck zum Anfang \\ 
\hline 
13 & 1, 5, 11 & 1 wird ausgew\"ahlt \\ 
\hline 
14 & 1, 5, 11 & mit 5 vergleichen \\ 
\hline 
15 & 1, 5, 11 & $1<5\curvearrowright$ n\"achste Zahl \\ 
\hline 
16 & 1, 5, 11 & 5 wird ausgew\"ahlt \\ 
\hline 
17 & 1, 5, 11 & mit 11 vergleichen \\ 
\hline 
18 & 1, 5, 11 & $5<11\curvearrowright$ n\"achste Zahl \\ 
\hline 
19 & 1, 5, 11 & Ende der Reihe erreicht \\ 
\hline 
\end{tabular} 
\begin{table}[htpb]
  \centering
  \begin{tabular}{c c c c c c}
    bla & bla & bla \\
  \end{tabular}
  \caption{<++>}
\end{table}
<++>
\end{table}

\item Der Algorithmus arbeitet f\"ur jede m\"ogliche Zahlenreihe korrekt, da das Ende der Zahlenreihe nur erreicht wird, wenn die Zahlen bereits korrekt sortiert sind. Nach dem Vertauschen zweier Zahlen wird jeweils wieder vorne angefangen, sodass dadurch keine Probleme entstehen k\"onnen. Auch das doppelte auftauchen einer Zahl in der Reihe ist kein Problem, da der Algorithmus die beiden Zahlen, wenn sie direkt nacheinander auftauchen als richtig sortiert behandelt.
\end{enumerate}

\subsection*{Aufgabe 2}
\begin{enumerate}[a)]

\item Eine Variable ist benannter Beh\"alter f\"ur Werte eines bestimmten Datentyps. Den in einer Variable gespeicherten Wert kann man \"andern. Werte k\"onnen als Bin\"arzahlen dargestellt werden. Werte sind fest. Mit Werten kann man Operationen durchf\"uhren und Werte k\"onnen gespeichert werden.

\item Eine Zuweisung ist eine Operation, bei der einer Variable ein Wert zugeordnet wird. Dabei kann der Variable auch ein Ausdruck, der zun\"achst  berechnet werden muss zugewiesen werden. Ein Beispiel w\"are:
\begin{lstlisting}
x=x+1
\end{lstlisting}

\end{enumerate}
\subsection*{Aufgabe 3}
\begin{enumerate}[a)]
  \item Es hilft, eine Vorstellung f\"ur das Problem zu gewinnen
    Man ben\"otigt sie um im Nachfolgenden Stichprobenartige Tests
    der L\"osung des Problems durchzuf\"uhren.
  \item
    Es widerspricht dem DRY (''Don't Repeat yourself``) Paradigma.
  \item
    Das don't Repeat yourself Paradigma ist sehr interessant, auch das es auf
    Definition von Variablen angewendet werden kann.
\end{enumerate}

\end{document}
